\documentclass[12pt]{article}

% --------------------------------
%% Preamble
% --------------------------------
\usepackage[parfill]{parskip} % Remove paragraph indentation
\usepackage[margin=1in]{geometry}
\usepackage{array} % Required for boldface (\bf and \bfseries) tabular columns
%\pagestyle{empty} % Suppress page numbers
\usepackage{amsmath}

\usepackage{hyperref} %To add hyperlinks
\hypersetup{colorlinks=true,urlcolor=blue,citecolor=blue}

\usepackage{titlesec} %Formatting for section & subsection titles
%Format sections and spacing
\titleformat{\section}{\normalfont\bfseries}{\thesection}{1em}{\MakeUppercase}[\titlerule]
\titlespacing{\section}{0pt}{5.0pt}{0.5pt}
%Format subsections and spacing
\titleformat{\subsection}{\normalfont\itshape}{\thesubsection}{1em}{}
\titlespacing{\subsection}{0pt}{0.5pt}{0.5pt}

\usepackage{enumitem} %Formatting ordered lists
\setlist[itemize]{noitemsep, topsep=0pt}

\usepackage{fancyhdr} %Add fancy header
\pagestyle{fancy}
\fancyhead[L]{Sushobhan Sen}
\fancyhead[R]{Teaching Statement}



\begin{document} \sloppy %sloppy command handles some line overflows
\begin{center}
{\large \uppercase{\textbf{Teaching Statement}}}
\end{center}

Teaching and research are two key pillars of higher education. While universities have traditionally been divided on the basis of their relative focus on teaching and research, my teaching philosophy is to apply the same scientific principles used in research to teaching. I have had several opportunities to engage in teaching during my training, including two teaching fellowships that enabled me to teach a senior design class as the lead instructor, and two teaching certificates through which I was able to demonstrate my skills. I received good course evaluations as an instructor, and I believe that I will be able to hit the ground running as an assistant professor.

\section*{Teaching Philosophy and Experience}
As a teacher, I have relied on two tools, evidence-based learning and Problem-Based Learning (PBL), to improve learning outcomes. My support for evidence-based learning stems from the American Society for Engineering Education's (ASEE) call for engineering educators to make use of the scientific literature and implement innovative teaching strategies in the classroom. I have been a Teaching Assistant (TA) as well as the lead instructor of record for a senior design class on four separate occasions, with a class size of 50-70 students in each instance. This class involved both a traditional lecture format as well as a semester-long design project on roadway and drainage design, and while it was designed for seniors, it included juniors as well as many graduate students from a variety of social and academic backgrounds. My role in the class was broad, from designing the syllabus and project to conducting assessments and regularly interacting with students in smaller groups and office hours. 

I designed the course so that students would work in teams as consultants, with the instructor being the client, so that in addition to learning the technical material, they would also be exposed to a professional working environment that included regular consultations with the client as well as professionally documenting and reporting all their work and hours logged. To design and improve the class, I read and applied evidence-based techniques from pedagogical literature, including active learning techniques such as small group discussions, best practices for creating and managing student groups (including obtaining regular feedback and managing group dynamics), and using Bloom's Taxonomy to design lectures and assessments. The class was a mixture of in-person and online (asynchronous) learners, and I used best practices from literature to manage both of them successfully. Several of my students came from minority backgrounds and were first-generation college students, and I gained a breadth of experience in ensuring that they had a valuable educational experience in my classroom.

I have also used PBL in my classes to improve students' critical reasoning skills. In the aforementioned senior design class, this was achieved by introducing autonomous vehicles into roadway design, which required students to re-examine current design guidelines, update them as necessary, and implement them in their projects. Through this process, students were given an opportunity to take a critical look into design standards, which would equip them for real-world problems where significant engineering judgment would be required. As part of a fellowship, I also collected data on outcomes and student reactions, which showed that students both benefited from and enjoyed this approach. I have received excellent student feedback and have used it to continuously fine-tune my approach over the years. The material I developed for the course as well as the student feedback served as an important piece in the most recent ABET re-accreditation package for the Department of Civil and Environmental Engineering at the University of Illinois.

In addition to the senior design class, I was also a TA for inter-disciplinary freshman class on disaster damage assessment, in which I used PBL to teach students about assessing public infrastructure during a week-long field visit to Puerto Rico, six months after it was devastated by Hurricane Maria. In the future, I will continue to use evidence-based and problem-based learning to enhance student outcomes.

In addition to classroom teaching, I have also served as a mentor to several undergraduate students, including some from under-represented groups. One of them worked with me from 2015 to 2018, during which time I guided them through research over a variety of projects. As part of that training, I encouraged them take ownership of their project and come up with new ideas. This led my student to present a poster at the $11^{th}$ International Conference on Concrete Pavements, and co-author a peer-reviewed journal paper with me. Upon graduating in 2018, the student joined a major R-1 university for their MS, and it was my privilege to provide a recommendation letter towards their admission package, as their mentor for three years. Another student that I had mentored is currently earning a PhD at another major R-1 university. Later on, I was able to use my mentorship experience to mentor new graduate students across engineering disciplines as a TA for a class on teaching and leadership. These experiences will help me mentor new students, both graduate and undergraduate in senior design classes, graduate-level independent research, and doctoral research, including students from under-represented backgrounds.

\section*{Teaching Plan}
As a faculty member, I will teach a new introductory course on computing in civil engineering for freshmen, which I have already developed. For upperclassmen, I will teach classes on Transportation Engineering, Fluid Mechanics, and Heat Transfer, which will also encompass computational aspects of these fields. 

For graduate students, I will teach courses on Computational Fluid Dynamics (CFD) and numerical methods for engineering problems using a PBL approach. My focus will be on teaching students to develop solutions that would be useful for their own research and professional work. In addition, I will also develop a new course on Computational Analysis of Urban Systems, through which I will disseminate the tools and findings from my research. Throughout my career, I will stay abreast of emerging pedagogical innovations and apply them to my classroom.

\end{document}