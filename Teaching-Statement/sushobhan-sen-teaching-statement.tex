\documentclass[12pt]{article}

% --------------------------------
%% Preamble
% --------------------------------
\usepackage[parfill]{parskip} % Remove paragraph indentation
\usepackage[margin=1in]{geometry}
\usepackage{array} % Required for boldface (\bf and \bfseries) tabular columns
%\pagestyle{empty} % Suppress page numbers
\usepackage{amsmath}
\usepackage{soul}

\usepackage{hyperref} %To add hyperlinks
\hypersetup{colorlinks=true,urlcolor=blue,citecolor=blue}

\usepackage{titlesec} %Formatting for section & subsection titles
%Format sections and spacing
\titleformat{\section}{\normalfont\bfseries}{\thesection}{1em}{\MakeUppercase}[]
\titlespacing{\section}{0pt}{5.0pt}{0.5pt}
%Format subsections and spacing
\titleformat{\subsection}{\normalfont\itshape}{\thesubsection}{1em}{}
\titlespacing{\subsection}{0pt}{0.5pt}{0.5pt}

\usepackage{enumitem} %Formatting ordered lists
\setlist[itemize]{noitemsep, topsep=0pt}

\usepackage{fancyhdr} %Add fancy header
\pagestyle{fancy}
\fancyhead[L]{Sushobhan Sen}
\fancyhead[R]{Teaching Statement}



\begin{document} \sloppy %sloppy command handles some line overflows
\begin{center}
{\large \uppercase{\textbf{Teaching Statement}}}
\end{center}

Teaching and research are two key pillars of higher education. While the Carnegie Classification of Institutions of Higher Education classifies universities based on their relative focus on teaching and research \cite{carnegie1994classification}, from my perspective, both of these are part of the academic activity of creating and disseminating knowledge. I believe that teaching can be enhanced through research to provide a fulfilling education to students and foster the next generation of problem-solvers.

\section*{Teaching Philosophy}
\subsection*{Pedagogical}
My pedagogy is based on two pillars: evidence-based learning and Problem-Based Learning (PBL). My support for evidence-based learning stems from the American Society for Engineering Education's (ASEE) call for engineering educators to make use of the scientific literature and implement innovative teaching strategies \cite{jamieson2009creating}. In my classroom, I will implement the latest pedagogical research while also collecting data to gauge its effectiveness. Through this, I hope to continuously improve as an educator and raise my students' learning outcomes.

PBL has several definitions, but at its heart is that students should lead their own learning while the teacher acts as an enabler \cite{albanese1993problem}. This is particularly conducive to today's world, where data and tools are readily available and students need to be equipped with the skills to learn them. I will implement PBL in my classes by assigning one or more projects during the semester, which they will solve using tools taught in class. Lectures will be built to supplement the projects, enabling students to immediately apply skills to a realistic problem. My lectures will use active learning so that students think about their projects even as they gain new information.

\subsection*{Technical}  
My teaching philosophy will be an extension of my research philosophy. I believe that a modern education should be inter-disciplinary, and the breadth of classes that I teach will reflect that. Through the PBL approach, I will give students an understanding of where their knowledge stands in a larger engineering context, and how skills from other disciplines could be applied to their projects. Specifically, I am looking towards integrating Civil Engineering, Mechanical Engineering, and Computer Science. In addition, as evidenced by student and employer demands \cite{magana2012motivation}, computational literacy is critical in the $21^{st}$ century. In my classes, I will teach students to use commercial and open-source software. I will also incorporate a programming component into my courses. In my teaching portfolio, I will have classes where I teach at least one scripting language (Python or MATLAB) or one compiled language (C or C++), which students will use for their projects. 

\section*{Teaching Experience}
As a student, I have had an enriching, multi-year teaching experience for a senior design class on the Geometric Design of Roads with upperclassmen and graduate students. I was a Teaching Assistant (TA) for this class in 2016 (68 students) and 2018 (62 students), and also the instructor in 2017 (55 students) as a teaching fellow. In these engagements, I spearheaded a PBL approach to the class by designing projects that incorporated Autonomous Vehicles (AVs) into the road design, and teaching students to implement their designs on AutoCAD Civil3D. In the 2018 engagement, I conducted a teaching effectiveness study to demonstrate how this approach had led to students developing new design guidelines for AVs, improving their meta-cognition and ability to work in teams.

In 2018, I was a co-instructor for a Disaster Relief Projects class that conducted an infrastructure assessment in Puerto Rico, six months after Hurricane Maria. This class had 37 undergraduate students, ranging from freshmen to seniors, from a variety of departments. I used the infrastructure assessment as a project to teach students to apply modern imaging tools to catalog infrastructure damage, develop sustainable solutions, and document their findings in a report to local communities. Through this experience, I was able to harness the strengths of an inter-disciplinary class for developing real-world solutions. 

In addition, I have been a mentor to an undergraduate student from 2015 to 2018, during which time I guided them through research over a variety of projects. As part of that training, I encouraged them take ownership of their project and come up with new ideas. Through this process, my student presented a poster at the $11^{th}$ International Conference on Concrete Pavements, and co-authored a paper with me at the $96^{th}$ Annual Meeting of the Transportation Research Board. We have also submitted a co-authored journal paper, which is under review. Upon graduating in 2018, the student joined a major R-1 university for their MS, and it was my privilege to provide a recommendation letter, as their mentor of three years.

During my time at the University of Illinois, I was extremely satisfied by a graduate-level course on Numerical Fluid Dynamics that I took in the Department of Atmospheric Sciences. The class elegantly combined fluid mechanics and computational science through a PBL-approach and supplemented my research. I used this class as a template to develop an introductory course on computing in civil engineering as a project for a class on Teaching and Leadership that I took. I will use this experience to structure my future classes around PBL and computing. 

\section*{Plans for the Future}
As a faculty member, I will implement an introductory course on computing in civil engineering for freshmen, which I have already developed. For upperclassmen, I will teach classes on Transportation Engineering, Fluid Mechanics, and Heat Transfer, including computational aspects of those fields. For graduate students, I will develop PBL courses on Computational Fluid Dynamics (CFD) and numerical methods for engineering problems, as well as a class on Urban Environmental Modeling that will disseminate findings from my research. I will use evidence-based learning to continuously improve these classes. Finally, I also look forward to mentoring graduate and undergraduate students to conduct cutting-edge research.

\bibliography{teaching-statement-bibliography} 
\bibliographystyle{ieeetran}

\end{document}