\documentclass[12pt]{article}

% --------------------------------
%% Preamble
% --------------------------------
\usepackage[parfill]{parskip} % Remove paragraph indentation
\usepackage[margin=1in]{geometry}
\usepackage{array} % Required for boldface (\bf and \bfseries) tabular columns
%\pagestyle{empty} % Suppress page numbers
\usepackage{amsmath}

\usepackage{hyperref} %To add hyperlinks
\hypersetup{colorlinks=true,urlcolor=blue,citecolor=blue}

\usepackage{titlesec} %Formatting for section & subsection titles
%Format sections and spacing
\titleformat{\section}{\normalfont\bfseries}{\thesection}{1em}{\MakeUppercase}[\titlerule]
\titlespacing{\section}{0pt}{5.0pt}{0.5pt}
%Format subsections and spacing
\titleformat{\subsection}{\normalfont\itshape}{\thesubsection}{1em}{}
\titlespacing{\subsection}{0pt}{0.5pt}{0.5pt}

\usepackage{enumitem} %Formatting ordered lists
\setlist[itemize]{noitemsep, topsep=0pt}

\usepackage{fancyhdr} %Add fancy header
\pagestyle{fancy}
\fancyhead[L]{Sushobhan Sen}
\fancyhead[R]{Research Statement}



\begin{document} \sloppy %sloppy command handles some line overflows

\begin{center}
{\large \uppercase{\textbf{Research Statement}}}
\end{center}

According to the World Bank \cite{Urbanpop19:online}, about a decade ago, the proportion of people in the world living in urban areas crossed those living in rural areas for the first time in modern history, and continues to rise steadily, with a majority of the increase coming from developing countries. This surge of humanity moving into urban areas has been the subject of studies in disciplines ranging from engineering to anthropology, all trying to understand how cities work and how they can be made to work better. In his seminal work, \textit{A Theory of Good City Form}, Kevin Lynch \cite{lynch1984good} discussed the impact of urban form on its inhabitants, outlining five performance indicators on which to judge a good city: vitality, sense, fit, access, and control. His work has inspired researchers and planners to describe a city as a machine \cite{molotch1976city, mcfarlane2011city, lloyd2001city, oke1973city}, with specific properties and functions that can be optimized.

My research, inspired by Lynch, looks at a city as a machine with three interdependent parts: infrastructure, environment, and people. In my doctoral studies, I looked at the relationship between infrastructure (specifically, construction materials and pavements) and the local microclimate. In the future, I intend to grow beyond that by bringing together a variety of disciplines, such as mechanical engineering and computer science, under one lab to study each of these parts separately, as well as the relationship between them. As more people live in cities, my research will look to solve existing and emerging problems of heat islands, pollution, noise, and resource scarcity, to make cities better and more resilient places to live in.

\section*{Previous Work}
For my Master's thesis and doctoral dissertation, I investigated the relationship between urban infrastructure and the environment through a mixture of field and laboratory testing and computational modeling. I measured the thermal and optical properties of pavement materials, for which I developed a new technique to measure the reflectance of asphalt and concrete specimens in the field using an albedometer \cite{sen2018albedo}, backed by laboratory measurements using a spectrophotometer. I developed one of the first models on the aging behavior of asphalt reflectance \cite{sen2016aging} taking local conditions into account, and proposed a technique to calibrate the model easily at any location. I investigated the variation of the thermal and optical properties of asphalt and concrete specimens with a variety of mix designs using a rapid laboratory technique.

Using the thermal and optical properties, I investigated the impact of pavements on the local urban microclimate by using Computational Fluid Dynamics (CFD) to resolve the wind and temperature fields within a city block. To do that, I developed numerical models that resolve the temperature field inside a pavement \cite{sen2017microscale} and coupled that to an urban canopy CFD model \cite{sen2017uncoupled}. I demonstrated that, in addition to the thermal and optical properties of the pavement, the configuration of the urban form around the pavement, as well as environmental conditions on a larger scale, affect the canopy-level air temperature above the pavement. This, in turn, affects human health and comfort and is an important measure of the performance of a city. My research showed that there is an inherent relationship between infrastructure and the urban environment, and a unified approach to study the two is necessary.

\section*{Future Work}
In my lab, I will further expand my research to comprehensively study the properties and performance of urban infrastructure, including pavements, walls, roofs, water bodies, and vegetation, and the relationships between them. Continuing from previous studies that used laboratory techniques or remote sensing to examine these \cite{berdahl1997preliminary, niachou2001analysis, sun2012can, gallo1993use}, my lab will study these surfaces in-situ at a small scale using Internet of Things (IoT)-based sensors mounted on the ground, cars, and drones. Here, I am inspired by the city of Baltimore's Urban Heat Island sensor array. My lab will develop and deploy sensors with the aim of measuring meteorological variables around specific urban blocks to study its effect on the surrounding microclimate. Such a study is useful for engineers to plan individual projects with the aim of mitigating the impact on the surrounding environment. 

Further, my lab will use field measuremets to develop and validate models of the urban environment, using both computational and experimental approaches. Computationally, it is difficult to resolve turbulence in the wake of a building, and more so if other buildings are present in the wake itself. An additional challenge is creating high quality meshes of urban areas given the irregular geometries involved. Most urban CFD researchers use Reynold's Average Navier-Stokes (RANS) modeling with turbulence models that do not sufficiently resolve wake turbulence \cite{blocken2015computational}. My lab will look to develop CFD codes that specifically address this issue using alternative but more-expensive methods such as Large Eddy Simulation (LES) and Direct Numerical Simulation (DNS), and then use machine learning to develop less-expensive turbulence models that work with a RANS approach for urban flows. We will also use machine learning to refine meshes to capture physics of interest. We will make significant use of supercomputers for this. With these tools, we will investigate heat, humidity, noise and pollutant distribution in cities during extreme events, such as congestion hours on roads and heat waves, and develop solutions for cities to better handle such events.

To study the urban environment experimentally, my lab will perform Wind Tunnel (WT) studies. For this, we will develop a wind tunnel that is capable of generating a stratified boundary layer flow. This WT can then be used to study flow behavior in a city during extreme events. A common challenge with urban WT testing is the difficulty in creating models that capture fine geometric features. To address this, my lab will look to use 3D printing for generating scaled models. A major aim will be to develop wind, heat, noise and pollution safety guidelines for designing urban infrastructure to mitigate their environmental impact. Wind tunnel experiments will also be used to fine-tune CFD modeling. 

Ultimately, cities have to be designed for the people that inhabit them, and engineering designs must optimally use resources to benefit the largest number of people. My lab will look to identify how people in a city behave in response to urban infrastructure and environment, and delineate areas where the largest number of people are at risk to extreme environmental effects. We will do this using both traditional survey techniques as well as computational tools in the digital social sciences. By locating areas that require improvements, we will use the insight gained from the lab to design solutions that fit each case best. 

\section*{Funding and Students}
As my research area will cover a number of disciplines, several funding agencies are on my radar. I will apply to NASA's Research Opportunities in Space and Earth Sciences (ROSES) program, the Environmental Protection Agency's Health, Sustainability, and Air Research Grants, and the Department of Energy's Office of Science and ARPA-E for grants on studying urban infrastructure and environment. The National Science Foundation's Divisions of Chemical, Bioengineering, Environmental, and Transport Systems and Civil, Mechanical and Manufacturing Innovation, as well as their Critical Resilient Interdependent Infrastructure Systems and Processes (CRISP) and Big Data programs are other funding sources that I will be applying to. For behavior-related aspects of my research, I will apply to the Social Science Research Council's (SSRC) Abe Fellowship and the Russel Sage Foundation together with collaborators from the social sciences. For supercomputer resources, I will apply to the XSEDE network and Amazon Web Services (AWS) research grants. 

Finally, I will apply for an NSF CAREER award. I believe that because of the multi-disciplinary nature of my research and its strong impact on urban societies, I will be quite competitive for the award. The award will give me the opportunity to advance urban studies as an inter-disciplinary area of study to provide solutions for the growing urban population around the world.  

My long-term aim is to establish an inter-disciplinary Center for Urban Studies to institutionalize my research and attract other scholars to it. In my early career, I will interact with and advise students from several disciplines, specifically Mechanical Engineering, Civil Engineering, Urban Planning, and Computer Science. I will work towards obtaining the necessary departmental affiliations to do that. I will also collaborate with faculty and students from Electronics Engineering, Cultural Anthropology, Geography, and Atmospheric Sciences. I believe that by having students from such a wide variety of areas in one lab, they will have a unique and fulfilling research experience that will prepare them to meet the complex challenges of future cities. I would also welcome undergraduate students to contribute to my lab as valued members.

\bibliography{research-statement-bibliography} 
\bibliographystyle{unsrt}

\end{document}