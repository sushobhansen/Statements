\documentclass[12pt]{article}

% --------------------------------
%% Preamble
% --------------------------------
\usepackage[parfill]{parskip} % Remove paragraph indentation
\usepackage[margin=1in]{geometry}
\usepackage{array} % Required for boldface (\bf and \bfseries) tabular columns
%\pagestyle{empty} % Suppress page numbers
\usepackage{amsmath}

\usepackage{hyperref} %To add hyperlinks
\hypersetup{colorlinks=true,urlcolor=blue,citecolor=blue}

\usepackage{titlesec} %Formatting for section & subsection titles
%Format sections and spacing
\titleformat{\section}{\normalfont\bfseries}{\thesection}{1em}{\MakeUppercase}[\titlerule]
\titlespacing{\section}{0pt}{5.0pt}{0.5pt}
%Format subsections and spacing
\titleformat{\subsection}{\normalfont\itshape}{\thesubsection}{1em}{}
\titlespacing{\subsection}{0pt}{0.5pt}{0.5pt}

\usepackage{enumitem} %Formatting ordered lists
\setlist[itemize]{noitemsep, topsep=0pt}

\usepackage{fancyhdr} %Add fancy header
\pagestyle{fancy}
\fancyhead[L]{Sushobhan Sen}
\fancyhead[R]{Research Statement}



\begin{document} \sloppy %sloppy command handles some line overflows

\begin{center}
{\large \uppercase{\textbf{Research Statement}}}
\end{center}

``Like a piece of architecture, the city is a construction in space, but one of vast scale, a thing perceived only in the course of long spans of time'' - Kevin Lynch, \textit{The Image of the City }

According to the World Bank, in 2008, the proportion of people in the world living in urban areas crossed those living in rural areas for the first time in modern history, and continues to rise steadily to over 4 billion people today. This surge of humanity into cities has led to an unsustainable increase in the consumption of natural resources and a decline in quality of life. For example, the OECD estimates that urban air pollution could lead to 6-9 million excess deaths by 2060 in industrialized nations alone, with an annual economic impact of \$2.6 trillion. Elevated temperatures, pollution, congestion, noise, water scarcity, and socio-economic disparities are manifestations of this challenge. In the future, the most pressing issue for civil engineers and urban planners will be to re-design cities to be smart, resilient, and sustainable. As a researcher on the cutting edge of this effort, my research philosophy is to study and design cities as a machine with three interconnected parts: infrastructure, environment, and people. My research will take advantage of traditional computational tools, modern advances in machine learning, and Internet of Things (IoT) to design smart cities around these three parts so that they are sustainable and resilient to the challenges of the next century.

\section*{Research Plan}
In my previous work, I have used a combination of materials characterization, in-situ sensors, and urban Computational Fluid Dynamics (CFD) to study the interaction between urban form, construction materials, and urban microclimate, which has a significant impact on public health and energy consumption. My work has shown that urban areas can be engineered to be cooler through targeted engineering interventions, and that computational methods can be a useful tool to determine the best strategy to maximize the benefits across the city. This was a new area of research within the research group, which I built from the ground-up during my MS and PhD, producing several papers and delivering invited talks at a Transportation Research Board (TRB) webinar, a seminar at Carnegie Mellon University, and presentations at several international conferences.

As an assistant professor, I will expand my research to develop smart engineering interventions to make cities more livable and resilient to the negative effects of urban sprawl and climate change. Some of these interventions include reflective surfaces, urban forestry, pervious pavements, urban water bodies, green roofs, and green walls. While previous studies have found these to be generally beneficial, the inherent complexity of urban areas due to changes in land use and land cover over very short distances leads to a lot of uncertainty. My research will aim to quantify these benefits while taking this uncertainty into account, leading to a smart city that is aware of its unique environment and where targeted interventions can be made to channel resources towards vulnerable communities.

The first project towards this goal will be the development and deployment of a network of sensors within the city of El Paso to measure in-situ temperature, humidity, pollution, and noise. While limited data for some of these variables is available from public and private agencies, a smart city requires more high-quality data collected at high spatial and temporal resolutions to enable decision-making. The project will seek to develop low-cost, ground-based sensors using off the shelf micro-controllers, electronic sensors, and 3D printed parts connected through IoT, which also has the potential for generating patents. These sensors will first be deployed within a neighborhood and eventually be scaled up to the entire city. In addition to this effort, other projects will look to use mobile sensors, remote sensing, atmospheric reanalyses, and drones to collect more data.

The high-quality data collected will set the stage for further projects to develop and validate models of the urban environment using both data science as well as computational mechanics. The large volume of data collected will be used to train machine learning models to determine the complex relationship between infrastructure and the environment, and will then be used to quantify the benefits of engineering interventions, taking into account uncertainties in both space and time. Another approach will be to develop and use calibrated physics-based computational models of the urban environment using the collected data. By combining both data-based and physics-based approaches with high-quality data collection, my research would be at the very cutting edge of smart cities.

\section*{Funding, Collaboration and Students}
As my research area will cover a number of disciplines, several funding agencies are on my radar. I will apply to NASA's Research Opportunities in Space and Earth Sciences (ROSES) program, the Environmental Protection Agency's Health, Sustainability, and Air Research Grants, and the Department of Energy's Office of Science and ARPA-E for grants on studying urban infrastructure and environment. The National Science Foundation's Divisions of Chemical, Bioengineering, Environmental, Computer and Network Systems, and Transport Systems and Civil, Mechanical and Manufacturing Innovation, as well as their Long-Term Ecological Research (LTER) and Big Data programs are other funding sources that I will be applying to. For computing resources, I will apply to the XSEDE network and Amazon Web Services (AWS) research grants. Finally, I will apply for an NSF CAREER award.  

In my lab, I will interact with and advise students from several disciplines, specifically Mechanical Engineering, Civil Engineering, Urban Planning, and Computer Science. I believe that by having students from such a wide variety of areas in one lab, they will have a unique and fulfilling research experience that will prepare them to meet the complex challenges of future cities. I would also welcome undergraduate students to contribute to my lab as valued members.

\end{document}