\documentclass[12pt]{article}

% --------------------------------
%% Preamble
% --------------------------------
\usepackage[parfill]{parskip} % Remove paragraph indentation
\usepackage[margin=1in]{geometry}
\usepackage{array} % Required for boldface (\bf and \bfseries) tabular columns
%\pagestyle{empty} % Suppress page numbers
\usepackage{amsmath}

\usepackage{hyperref} %To add hyperlinks
\hypersetup{colorlinks=true,urlcolor=blue,colorlinks=false}

\usepackage{titlesec} %Formatting for section & subsection titles
%Format sections and spacing
\titleformat{\section}{\normalfont\bfseries}{\thesection}{1em}{\MakeUppercase}[]
\titlespacing{\section}{0pt}{5.0pt}{0.5pt}
%Format subsections and spacing
\titleformat{\subsection}{\normalfont\itshape}{\thesubsection}{1em}{}
\titlespacing{\subsection}{0pt}{0.5pt}{0.5pt}

\usepackage{enumitem} %Formatting ordered lists
\setlist[itemize]{noitemsep, topsep=0pt}

\usepackage{fancyhdr} %Add fancy header
\pagestyle{fancy}
\fancyhead[L]{Sushobhan Sen}
\fancyhead[R]{Teaching Statement}



\begin{document} \sloppy %sloppy command handles some line overflows
\begin{center}
{\large \uppercase{\textbf{Teaching Statement}}}
\end{center}

In a traditional university system, teaching and research are often seen as two separate activities that a faculty member is expected undertake. Indeed, the Carnegie Classification of Institutions of Higher Education classifies universities based on their relative focus on teaching and research \cite{carnegie1994classification}. However, from my perspective, both of these are part of the academic activity of creating and disseminating knowledge. Therefore, I believe that a good researcher can be a good teacher, and the tools of inquiry that are applied to research can equally be applied to teaching. 

In the following paragraphs, I hope to highlight my approach to teaching through the lens of a researcher by outlining my teaching philosophy and demonstrating how I have applied it so far in the classroom. I will then conclude by outlining my plans for running classrooms as an independent scholar and teacher. 

\section*{Teaching Philosophy}


\bibliography{teaching-statement-bibliography} 
\bibliographystyle{unsrt}

\end{document}