\documentclass[12pt]{article}

% --------------------------------
%% Preamble
% --------------------------------
\usepackage[parfill]{parskip} % Remove paragraph indentation
\usepackage[margin=1in]{geometry}
\usepackage{array} % Required for boldface (\bf and \bfseries) tabular columns
%\pagestyle{empty} % Suppress page numbers
\usepackage{amsmath}
\usepackage{soul}

\usepackage{hyperref} %To add hyperlinks
\hypersetup{colorlinks=true,urlcolor=blue,citecolor=blue}

\usepackage{titlesec} %Formatting for section & subsection titles
%Format sections and spacing
\titleformat{\section}{\normalfont\bfseries}{\thesection}{1em}{\MakeUppercase}[]
\titlespacing{\section}{0pt}{5.0pt}{0.5pt}
%Format subsections and spacing
\titleformat{\subsection}{\normalfont\itshape}{\thesubsection}{1em}{}
\titlespacing{\subsection}{0pt}{0.5pt}{0.5pt}

\usepackage{enumitem} %Formatting ordered lists
\setlist[itemize]{noitemsep, topsep=0pt}

\usepackage{fancyhdr} %Add fancy header
\pagestyle{fancy}
\fancyhead[L]{Sushobhan Sen}
\fancyhead[R]{Teaching Statement}



\begin{document} \sloppy %sloppy command handles some line overflows
\begin{center}
{\large \uppercase{\textbf{Teaching Statement}}}
\end{center}

Teaching and research are two key pillars of higher education. While the Carnegie Classification of Institutions of Higher Education classifies universities based on their relative focus on teaching and research \cite{carnegie1994classification}, I have worked towards enhancing my teaching using the same scientific approach and inter-disciplinary philosophy as I use for research.

\section*{Teaching Experience and Philosophy}
As a teacher, I have relied on evidence-based learning and Problem-Based Learning (PBL) to improve learning outcomes. My support for evidence-based learning stems from the American Society for Engineering Education's (ASEE) call for engineering educators to make use of the scientific literature and implement innovative teaching strategies \cite{jamieson2009creating}. My teaching experience includes a senior design class of 50-70 students on the geometric design of roads with upperclassmen and graduate students. I was a Teaching Assistant (TA) for this class in 2016 and 2018, and the instructor of record in 2017 as a teaching fellow. During these tenures, I read and applied techniques from pedagogical literature, including active learning, creating and managing student groups, and using Bloom's Taxonomy to design lectures and assessments. During this time, I also earned two teaching certificates to further develop my pedagogical skills. 

I have also used PBL in my classes to enable students to lead their own learning \cite{albanese1993problem}. In my geometric design class, I spearheaded a PBL approach to the class by designing projects that incorporated Autonomous Vehicles (AVs) into the road design, and teaching students to implement their designs in their class projects. In the 2018 engagement, I conducted a study on the ability of this approach to improve critical thinking, and a paper from that is currently under peer review. In 2018, I was also a TA for an inter-disciplinary class on disaster damage assessment, in which I used PBL to teach students about assessing public infrastructure during a week-long field visit to Puerto Rico, six months after it was devastated by Hurricane Maria. Following the survey, I worked with them to analyze and report their findings. In the future, I will continue to use evidence-based and problem-based learning to enhance student outcomes, and conduct studies on my own teaching to improve my approach.

In addition to the pedagogical aspects of my teaching, I have also put a strong emphasis on inter-disciplinary learning and teaching the latest technology. In my geometric design class, my students learned about different technologies used for autonomous vehicles, and were taught to implement their designs on AutoCAD Civil 3D. In my infrastructure assessment class, my students used drones, cameras, and a phone-based app to collect data. Evidence shows that both students and employers believe that computational literacy is critical in the $21^{st}$ century \cite{magana2012motivation}. Therefore, my classes in the future will continue to teach technologies from a variety of disciplines, including commercial and open-source software and programming languages, to solve problems efficiently. 

In addition to classroom teaching, I have also served as a mentor to an undergraduate student from 2015 to 2018, during which time I guided them through research over a variety of projects. As part of that training, I encouraged them take ownership of their project and come up with new ideas. This led my student to present a poster at the $11^{th}$ International Conference on Concrete Pavements, and co-author a paper with me at the $96^{th}$ Annual Meeting of the Transportation Research Board. We have also submitted a co-authored journal paper, which is under review. Upon graduating in 2018, the student joined a major R-1 university for their MS, and it was my privilege to provide a recommendation letter, as their mentor of three years. Later in 2018, I was able to use this experience to mentor new graduate students across engineering disciplines as a TA for a class on teaching and leadership. These experiences will help me mentor new students, both graduate and undergraduate, in the future.

Finally, as a student at the University of Illinois, I was extremely satisfied by a graduate-level course on Numerical Fluid Dynamics that I took in the Department of Atmospheric Sciences. The class elegantly combined fluid mechanics and computational science through a PBL-approach and supplemented my research. I used this class as a template to develop an introductory course on computing in civil engineering as a project for another class that I took. I will use this experience to develop and structure courses that I will teach in the future.  

\section*{Plans for the Future}
As a faculty member, I will teach an introductory course on computing in civil engineering for freshmen, which I have already developed. For upperclassmen, I will teach classes on Transportation Engineering, Fluid Mechanics, and Heat Transfer, which will also encompass computational aspects of these fields, as I outlined in my teaching philosophy. 

For graduate students, I will teach courses on Computational Fluid Dynamics (CFD) and numerical methods for engineering problems using a PBL approach. My focus will be on teaching students to develop solutions that would be useful for their own research and professional work. In addition, I will also develop a new course on Computational Analysis of Urban Systems, through which I will disseminate tools and findings from my research.

\bibliography{teaching-statement-bibliography} 
\bibliographystyle{ieeetran}

\end{document}