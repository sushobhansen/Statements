\documentclass[12pt]{article}

% --------------------------------
%% Preamble
% --------------------------------
\usepackage[parfill]{parskip} % Remove paragraph indentation
\usepackage[margin=1in]{geometry}
\usepackage{array} % Required for boldface (\bf and \bfseries) tabular columns
%\pagestyle{empty} % Suppress page numbers
\usepackage{amsmath}
\usepackage{soul}

\usepackage{hyperref} %To add hyperlinks
\hypersetup{colorlinks=true,urlcolor=blue,citecolor=blue}

\usepackage{titlesec} %Formatting for section & subsection titles
%Format sections and spacing
\titleformat{\section}{\normalfont\bfseries}{\thesection}{1em}{\MakeUppercase}[]
\titlespacing{\section}{0pt}{5.0pt}{0.5pt}
%Format subsections and spacing
\titleformat{\subsection}{\normalfont\itshape}{\thesubsection}{1em}{}
\titlespacing{\subsection}{0pt}{0.5pt}{0.5pt}

\usepackage{enumitem} %Formatting ordered lists
\setlist[itemize]{noitemsep, topsep=0pt}

\usepackage{fancyhdr} %Add fancy header
\pagestyle{fancy}
\fancyhead[L]{Sushobhan Sen}
\fancyhead[R]{Diversity Statement}



\begin{document} \sloppy %sloppy command handles some line overflows
\begin{center}
{\large \uppercase{\textbf{Diversity Statement}}}
\end{center}

\section*{Background}
Diversity has been a part of my life from the very beginning. Growing up in a military family in India, I have always had to move every few years to new places, and was thus perpetually the 'new guy' anywhere I went. Later on an an undergraduate student, I visited Germany and Japan, and then came to the US for graduate education. Thus, I continued my journey of discovering new places and people even beyond childhood and teenage years. Experiencing new and unfamiliar situations every few years made me think deeply about my fears and anxieties and how I could overcome them. Through this experience, I can relate to students from minority and under-represented groups, who also often find themselves in unfamiliar and often intimidating situations. 

As a result of my experiences, I was exposed to the many ways in which people and organizations manage diversity. I am particularly inspired by the saying, \textit{'Let noble thoughts come to us from all side,'} from the ancient Indian text, the Rig Veda (1.89.1). My philosophy towards diversity can be summarized in one word: \textit{listen}. I listen to people and observe things around me without being judgmental, putting myself in the situations I see around me in order to understand it. This helps me understand and empathize with people and be conscientious in my actions.

\section*{Experience}
As a result of my lifelong experience with diversity, I have participated in several outreach activities in order to give back to the community around me. As an undergraduate student, I spent a year teaching disadvantaged children in India basic arithmetic and reading. I also coached younger students in college debating and public speaking, helping them acquire valuable communication skills. It was these experiences that showed me that I enjoyed teaching and mentoring, which I have now turned into a career.

In graduate school, I participated in our annual Open House every year, where I taught K-12 students about mixing concrete and understanding how traffic is studied and managed. I also participated in professional student organizations to spread awareness about professional issues in civil engineering among the student community. Outside of an academic setting, I served as president of the Indian Graduate Students Association (IGSA) at my university. My activities in IGSA have been two-fold: first, I supported new graduate students from India to acclimatize and settle in to life in the US; and second, I spread awareness about the history and culture of India to members of the local community, helping them appreciate how minority cultures, like that of Indian-Americans, have contributed to the US. Towards this goal, I have also collaborated with other Asian-American organizations on campus, pooling together shared experiences and resources. 

\section*{Commitment}
As in the rest of my life, I am committed to encouraging and facilitating diversity in my research and teaching. I am fortunate that in both of these areas, I cover a broad range of fields and thus will be able to accommodate students coming in with diverse skills and backgrounds. In interacting with students, my principle will always be to listen to their concerns and help them find the tools they need to succeed.

I am acutely aware that my field of engineering is infamous for its lack of diversity, whether it be minority groups or women. Through my lifelong experience with diversity and my efforts to give back to the community, I hope to be part of the generation that, however slowly, changes that for the better. 

\end{document}