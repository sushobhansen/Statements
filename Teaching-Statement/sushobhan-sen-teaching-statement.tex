\documentclass[12pt]{article}

% --------------------------------
%% Preamble
% --------------------------------
\usepackage[parfill]{parskip} % Remove paragraph indentation
\usepackage[margin=1in]{geometry}
\usepackage{array} % Required for boldface (\bf and \bfseries) tabular columns
%\pagestyle{empty} % Suppress page numbers
\usepackage{amsmath}
\usepackage{soul}

\usepackage{hyperref} %To add hyperlinks
\hypersetup{colorlinks=true,urlcolor=blue,citecolor=blue}

\usepackage{titlesec} %Formatting for section & subsection titles
%Format sections and spacing
\titleformat{\section}{\normalfont\bfseries}{\thesection}{1em}{\MakeUppercase}[]
\titlespacing{\section}{0pt}{5.0pt}{0.5pt}
%Format subsections and spacing
\titleformat{\subsection}{\normalfont\itshape}{\thesubsection}{1em}{}
\titlespacing{\subsection}{0pt}{0.5pt}{0.5pt}

\usepackage{enumitem} %Formatting ordered lists
\setlist[itemize]{noitemsep, topsep=0pt}

\usepackage{fancyhdr} %Add fancy header
\pagestyle{fancy}
\fancyhead[L]{Sushobhan Sen}
\fancyhead[R]{Teaching Statement}



\begin{document} \sloppy %sloppy command handles some line overflows
\begin{center}
{\large \uppercase{\textbf{Teaching Statement}}}
\end{center}

In a traditional university system, teaching and research are often seen as two separate activities that a faculty member undertakes. Indeed, the Carnegie Classification of Institutions of Higher Education classifies universities based on their relative focus on teaching and research \cite{carnegie1994classification}. However, from my perspective, both of these are part of the academic activity of creating and disseminating knowledge. Therefore, I believe that a good researcher can be a good teacher, and the tools of inquiry that are applied to research can equally be applied to teaching. 

In the following paragraphs, I highlight my approach to teaching through the lens of a researcher by outlining my teaching philosophy and demonstrating how I have applied it so far in the classroom. I will then conclude by outlining my plans for running my classroom as an independent scholar and teacher. 

\section*{Teaching Philosophy}
\subsection*{Pedagogical}
The Merriam-Webster dictionary defines pedagogy as, "the art, science, or profession of teaching" \cite{merriam-webster-pedagogy}. Therefore, in explaining the pedagogical aspects of my teaching philosophy, I must explain its scientific basis. My pedagogy is based on two pillars: evidence-based learning and Problem-Based Learning (PBL). My support for evidence-based learning stems from the American Society for Engineering Education's (ASEE) call for engineering educators to make use of the scientific literature and implement innovative teaching strategies \cite{jamieson2009creating}. In my classroom, I will implement the latest pedagogical research while also collecting data to gauge its effectiveness. Through this, I hope to continuously improve as an educator and raise my students' learning outcomes.

PBL has several definitions, but at its heart is that students should lead their own learning while the teacher acts as an enabler \cite{albanese1993problem}. I will implement PBL in my classroom to encourage meta-cognition in my students and equip them with skills to solve new problems in the future. At the same time, I will also design my classes to take into account different scholastic capabilities of students, blending PBL with other pedagogical techniques in the process.    

\subsection*{Technical}  
From a technical perspective, my teaching philosophy will be an extension of my research philosophy. I believe that a modern education should be inter-disciplinary, and I look forward to teaching classes from different disciplines. Furthermore, I will focus on explaining the context of a class and how it is related to other classes that students might take. My aim will be to give students an understanding of where their knowledge stands in a larger engineering context, and how skills from other disciplines can complement their own. In addition, I believe that computational literacy is critical in the $21^{st}$ century. Indeed, research has shown that computational skills are demanded by students as well as employers \cite{magana2012motivation}. Therefore, my classes will focus on teaching students to use software as well as write good code to implement what they have learned.

\section*{Teaching Experience}
As a student, I have had an enriching, multi-year teaching experience for a 400-level class on the Geometric Design of Roads with upperclassmen and graduate students. I was a Teaching Assistant (TA) for this class in 2016 and 2018, and also the instructor in 2017 as a teaching fellow. In these engagements, I spearheaded a PBL approach to the class through an open-ended problem. At the same time, I used online tools to manage and collect regular feedback on the effectiveness of this class, and \hl{in 2018, I conducted a formal study to quantify this.}

In addition, I have been a mentor to an undergraduate student for three years, during which I was able to introduce them to research through a variety of projects. I encouraged them to read literature to understand the results of their experiments, and to publish them through co-authored papers and presentations at conferences. \hl{In 2018, they were admitted to a graduate program in a top university, and I was extremely happy to know that their work with me had been the inspiration for them to pursue a PhD.}

Finally, I was extremely satisfied with a class on Numerical Fluid Dynamics that I took at the University of Illinois. The class elegantly combined fluid mechanics and computational science through PBL. This class will serve as a template for me in the future.

\section*{Plans for the Future}
As a faculty member, I will develop a foundational course using PBL to introduce freshmen to computational techniques relevant to their discipline. For upperclassmen, I will teach classes on Pavement \& Traffic Engineering, Fluid Mechanics, and Heat Transfer, including computational aspects of those fields. For graduate students, I will develop PBL courses on Computational Fluid Dynamics (CFD) and numerical methods for engineering problems, and use an evidence-based approach to improve them. Finally, I look forward to mentoring graduate and undergraduate students to develop new, inter-disciplinary knowledge.

\bibliography{teaching-statement-bibliography} 
\bibliographystyle{ieeetran}

\end{document}