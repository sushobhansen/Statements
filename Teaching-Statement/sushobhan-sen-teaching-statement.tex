\documentclass[12pt]{article}

% --------------------------------
%% Preamble
% --------------------------------
\usepackage[parfill]{parskip} % Remove paragraph indentation
\usepackage[margin=1in]{geometry}
\usepackage{array} % Required for boldface (\bf and \bfseries) tabular columns
%\pagestyle{empty} % Suppress page numbers
\usepackage{amsmath}

\usepackage{hyperref} %To add hyperlinks
\hypersetup{colorlinks=true,urlcolor=blue,citecolor=blue}

\usepackage{titlesec} %Formatting for section & subsection titles
%Format sections and spacing
\titleformat{\section}{\normalfont\bfseries}{\thesection}{1em}{\MakeUppercase}[\titlerule]
\titlespacing{\section}{0pt}{5.0pt}{0.5pt}
%Format subsections and spacing
\titleformat{\subsection}{\normalfont\itshape}{\thesubsection}{1em}{}
\titlespacing{\subsection}{0pt}{0.5pt}{0.5pt}

\usepackage{enumitem} %Formatting ordered lists
\setlist[itemize]{noitemsep, topsep=0pt}

\usepackage{fancyhdr} %Add fancy header
\pagestyle{fancy}
\fancyhead[L]{Sushobhan Sen}
\fancyhead[R]{Teaching Statement}



\begin{document} \sloppy %sloppy command handles some line overflows
\begin{center}
{\large \uppercase{\textbf{Teaching Statement}}}
\end{center}

Teaching and research are two key pillars of higher education. While universities have traditionally been divided on the basis of their relative focus on teaching and research, my teaching philosophy is to apply the same scientific principles used in research to teaching. I have had several opportunities to engage in teaching during my training, including two teaching fellowships and two teaching certificates to develop my skills. 

\section*{Teaching Philosophy and Experience}
As a teacher, I have relied on evidence-based learning and Problem-Based Learning (PBL) to improve learning outcomes. My support for evidence-based learning stems from the American Society for Engineering Education's (ASEE) call for engineering educators to make use of the scientific literature and implement innovative teaching strategies. I have been a Teaching Assistant (TA) as well as an instructor-of-record for a senior design class on four separate occasions, with a class size of 50-70 students, including online participants. During these stints, I read and applied evidence-based techniques from pedagogical literature, including active learning, best practices for creating and managing student groups, and using Bloom's Taxonomy to design lectures and assessments. 

I have also used PBL in my classes to encourage more experiential learning. In the senior design class, I spearheaded an approach that used PBL to teach students about the potential impacts of widespread adoption of connected and autonomous vehicles on the design of roadway infrastructure. In addition, I was a TA for inter-disciplinary class on disaster damage assessment, in which I used PBL to teach students about assessing public infrastructure during a week-long field visit to Puerto Rico, six months after it was devastated by Hurricane Maria. In the future, I will continue to use evidence-based and problem-based learning to enhance student outcomes.

In addition to classroom teaching, I have also served as a mentor to an undergraduate student from 2015 to 2018, during which time I guided them through research over a variety of projects. As part of that training, I encouraged them take ownership of their project and come up with new ideas. This led my student to present a poster at the $11^{th}$ International Conference on Concrete Pavements, and co-author a peer-reviewed journal paper with me. Upon graduating in 2018, the student joined a major R-1 university for their MS, and it was my privilege to provide a recommendation letter towards their admission package, as their mentor for three years. Later in 2018, I was able to use this experience to mentor new graduate students across engineering disciplines as a TA for a class on teaching and leadership. These experiences will help me mentor new students, both graduate and undergraduate, in the future.

\section*{Teaching Plan}
As a faculty member, I will teach a new introductory course on computing in civil engineering for freshmen, which I have already developed. For upperclassmen, I will teach classes on Transportation Engineering, Fluid Mechanics, and Heat Transfer, which will also encompass computational aspects of these fields. 

For graduate students, I will teach courses on Computational Fluid Dynamics (CFD) and numerical methods for engineering problems using a PBL approach. My focus will be on teaching students to develop solutions that would be useful for their own research and professional work. In addition, I will also develop a new course on Computational Analysis of Urban Systems, through which I will disseminate the tools and findings from my research. Throughout my career, I will stay abreast of emerging pedagogical innovations and apply them to my classroom.

\end{document}