\documentclass[12pt]{article}

% --------------------------------
%% Preamble
% --------------------------------
\usepackage[parfill]{parskip} % Remove paragraph indentation
\usepackage[margin=1in]{geometry}
\usepackage{array} % Required for boldface (\bf and \bfseries) tabular columns
%\pagestyle{empty} % Suppress page numbers
\usepackage{amsmath}

\usepackage{hyperref} %To add hyperlinks
\hypersetup{colorlinks=true,urlcolor=blue,citecolor=blue}

\usepackage{titlesec} %Formatting for section & subsection titles
%Format sections and spacing
\titleformat{\section}{\normalfont\bfseries}{\thesection}{1em}{\MakeUppercase}[\titlerule]
\titlespacing{\section}{0pt}{5.0pt}{0.5pt}
%Format subsections and spacing
\titleformat{\subsection}{\normalfont\itshape}{\thesubsection}{1em}{}
\titlespacing{\subsection}{0pt}{0.5pt}{0.5pt}

\usepackage{enumitem} %Formatting ordered lists
\setlist[itemize]{noitemsep, topsep=0pt}

\usepackage{fancyhdr} %Add fancy header
\pagestyle{fancy}
\fancyhead[L]{Sushobhan Sen}
\fancyhead[R]{Research Statement}



\begin{document} \sloppy %sloppy command handles some line overflows

\begin{center}
{\large \uppercase{\textbf{Research Statement}}}
\end{center}

Data collected by the World Bank \cite{Urbanpop19:online} indicates that about a decade ago, the proportion of people in the world living in urban areas crossed those living in rural areas for the first time in history, and continues to rise steadily, with a majority of the increase coming from developing countries. This surge of humanity moving into urban areas has been the subject of studies in a variety of disciplines, all trying to understand how cities work and how they can be made to work better. In his seminal work, \textit{A Theory of Good City Form}, Kevin Lynch \cite{lynch1984good} discussed the impact of urban form on its inhabitants, outlining five performance indicators on which to judge a good city. His work has inspired researchers and planners to describe a city as a machine \cite{molotch1976city, mcfarlane2011city, lloyd2001city, oke1973city} with specific properties and functions.

My research, inspired by Lynch, looks at a city as a machine with three interdependent parts: infrastructure, environment, and people. In my doctoral studies, I looked at the relationship between infrastructure (specifically, pavements) and the local microclimate. As an independent researcher, I intend to grow beyond that by bringing together a variety of disciplines under one lab to study each of these parts separately, as well as the relationship between them. Furthermore, as an engineer, the aim of my lab will be to design each of these parts optimally to make the city a better place for people to live in. 

\section*{Previous Work}
For my Master's thesis and doctoral dissertation, I investigated the relationship between urban infrastructure and the environment. I measured the thermal and optical properties of pavement materials, for which I developed a new technique to measure the reflectance of asphalt and concrete specimens \cite{sen2018albedo}, developed one of the first models on the aging behavior of asphalt reflectance \cite{sen2016aging}, and investigated the variation of the thermal and optical properties of asphalt and concrete specimens with a variety of mix designs. In my research, I have used both field and laboratory techniques.

Using the thermal and optical properties, I investigated the impact of pavements on the local urban microclimate by using Computational Fluid Dynamics (CFD) to resolve the wind and temperature fields in a part of a city. To do that, I developed numerical models that resolve the temperature field inside a pavement \cite{sen2017microscale} and coupled that to an urban canopy CFD model \cite{sen2017uncoupled}. I demonstrated that, in addition to the thermal and optical properties of the pavement, the configuration of the urban form around the pavement, as well as environmental conditions on a larger scale, affect the canopy-level air temperature above the pavement. 

\section*{Future Work}
In my lab, I will further expand my research to study the properties and performance of urban infrastructure, including pavements, walls, roofs, water bodies, and vegetation. While each of these have been previously investigated to some extent using laboratory techniques or remote sensing \cite{berdahl1997preliminary, niachou2001analysis, sun2012can, gallo1993use}, my aim will be to study these surfaces in-situ at a small scale using Internet of Things (IoT)-based sensors. Here, I am inspired by the city of Baltimore's Urban Heat Island sensor array. However, my lab will develop and deploy sensors with the aim of measuring meteorological variables around \textit{specific} urban infrastructure (and not an entire city) to study its long-term effect on the surrounding microclimate. Such a study is useful for engineers to plan individual projects with the aim of mitigating the impact on the surrounding environment. 

Further, my lab will study the urban environment through both computational and experimental approaches. Computationally, it is difficult to resolve turbulence in the wake of a building and more so if other buildings are present in the wake itself. Most urban CFD researchers use RANS modeling with turbulence models that do not sufficiently resolve wake turbulence \cite{blocken2015computational}. My lab will look to develop CFD codes that specifically address this issue using alternative but more-expensive methods such as LES and DNS, and use machine learning to develop less-expensive turbulence models that work with a RANS approach for urban flows. We will make significant use of supercomputers for this. With these tools, we will investigate heat and pollutant distribution in cities during peak events, such as congestion hours on roads and heat waves, to design cities to better handle such events.

To study the urban environment experimentally, my lab will perform Wind Tunnel (WT) studies. Two challenges with urban WT studies is the difficulty in building prototypes for small-scale features such as vegetation and meeting similarity requirements for temperature and pollutant concentration. To address the former issue, my lab will use 3D printing to create detailed geometries for analysis, while to address the latter, we will develop best-practice guidelines to model the urban environment without meeting all similarity requirements. The final aim is to develop wind, heat, and pollution safety guidelines for designing urban infrastructure to mitigate their environmental impact. For both computational and experimental approaches, data collected using IoT-based sensors will be used to quantify the validity of these approaches.

Ultimately, cities have to be designed for the people that inhabit them, and engineering designs must optimally use resources to benefit the largest number of people. My lab will look to identify how people in a city behave in response to urban infrastructure and environment, and delineate areas where the largest number of people are at risk to extreme environmental effects. We will do this using both traditional survey techniques as well as data analytics of traffic and municipal data. By locating areas that require improvements, we will use the insight gained from the lab to design solutions that best fit each case. 

\section*{Funding and Students}
As my research area will cover a number of disciplines, several funding agencies are on my radar. I will apply to NASA, the Environmental Protection Agency, and the Department of Energy for grants on studying urban infrastructure and environment. The National Science Foundation (NSF) and the Department of Health \& Human Services (HHS) support research that benefits urban communities, and I will be applying to them as well. For behavior-related aspects of my research, I will apply to the Social Science Research Council (SSRC) and the National Endowment for the Humanities (NEH). For supercomputer resources, I will apply to the XSEDE network and the Amazon Web Services (AWS) research grant. Finally, I will apply for an NSF CAREER award.

In order to conduct inter-disciplinary research, I will train students from several disciplines, such as Mechanical Engineering, Civil Engineering, Electronics Engineering, Urban Planning, Cultural Anthropology, Computer Science, Aerospace Engineering, and Atmospheric Science. I will work towards obtaining the necessary departmental affiliations to do that. I believe that by having students from such a wide variety of areas in one lab, they will have a unique and fulfilling research experience that will prepare them to meet the complex challenges of future cities. I would also welcome undergraduate students to contribute to my lab as valued members. Ultimately, I aim towards establishing an inter-disciplinary Center for Urban Studies to institutionalize my research and attract other scholars to it.

\bibliography{research-statement-bibliography} 
\bibliographystyle{unsrt}

\end{document}